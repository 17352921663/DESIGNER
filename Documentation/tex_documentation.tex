\documentclass[12pt]{sop}
\title{Processing Study with Designer	}
\date{2/7/19}
\author{Dhiman}
\approved{TBA}
\begin{document}

\maketitle
\section{Introduction}
Desgner is a preprocessing pipeline that integrates commonly used dMRI preprocessing steps in a specific sequency to enhance SNR and minimize the use of smoothing.
Modifications were made to the original Designer script from NYU to incorporate additional parameterization such as KFA and MKT.
Our approach towards standardizing studies for reproducibility have allowed us to conclude that Designer is the best option with the following options:

This allows Designer to run its standard preprocessing techniques. To contrast with MUSC's pipeline, the table below highlights the preprocessings steps used and their order in both pipelines.

\begin{tabular}{ | c | r |}
Designer             & MUSC              \\
MP-PCA               & MP-PCA            \\
Gibb's Unringing     & Gibb'sUnringing   \\
Rigid-body alignment & Rician Correction	\\
EPI + eddy correction & Brain Mask \\
Brain Mask & EPI + Eddy \\
Smoothing & Smoothing \\
Rician correction & 
\end{tabular}

The primary advantages Designer has over MUSC's pipeline are rigid body alignment to aid eddy correction, and placement of Rician correction. In our testing of Rician correction placement, it was discovered that placing Rician correction at the end produced significantly different results than when it is placed after Gibb's unringing. Dr. Veerart's justification on the placement of Rician correction is as such:

\begin{quote}
"The rician bias correct using the methods of moments is poorly conditioned and depends on the precise (and accurate) estimation of the expectation value of the Rician distributed signals.  Applying this correction prior to motion and eddy current correction is most likely to be more accurate because signals varies, even within a local window. However, additional interpolation steps tend to smooth the data more, rendering a more precise estimation of the expectation value. At this stage, we consider that the precision gain outweighs the potential drop in accuracy due to the interpolation associated to the motion and EC distortion correction."
\end{quote}

Designer initiates preprocessing with MP-PCA noise estimationan and removal, followed by Gibb's unringing. Then, a 6-DOF rigid body alignmeny is performed to boost the performance of motion correction later on. Rigid body is followed by EPI distortion correction and eddy current correction, which also corrects for motion. A brain mask is then computed, followed by 1.25 FWHM smoothing at $5 \times 5$ kernel. This preprocessing pipeline ends with DT/KT estimation and parameterization.

\section{Amendments}

\section{Purpose}

\section{Scope}

\section{Installation}
Designer depends on the following software packages to run successfully:
\begin{enumerate}
\item MATLAB
\item Python 3.6
\item FSL
\item Mrtrix3
\end{enumerate}

Failing to meet any of these dependencies will prevent Designer from running. As of January 29, 2019, this pipeline was tested and certified to run on MATLAB 2018b and Python 3.6. Due to MATLAB's current support of only up to Python 3.6, version 3.7 has not been tested.

\subsection{Install FSL}
Obtain and install FSL from their official page at https://fsl.fmrib.ox.ac.uk/fsl/fslwiki/
Installation instructions for all compatible operating system are available on their website

\subsection{Install Mrtrix3}
Akin to FSL, installation instructions and downloads are avaible from MRtrix3's official page at http://www.mrtrix.org/
We found homebrew installation to be the most straightforward on Mac.

\subsection{Install MATLAB}
MATLAB is very straightforward to install because it's a packaged installer. Just run the installer, wait for it to finish, then proceed to Python configuration in the next section.

\subsection{Install Python}
There are several python packages available for installation for all operating systems. It is advisable to use a Conda-based distribution such as Miniconda or Anaconda. This guide is based on Miniconda for Max OS Mojave (v.10.14.3), so same directions apart from directory structures will apply.

\subsubsection{Download Miniconda}
Miniconda can be downloaded from their official page at Miniconda Latest Releases [https://conda.io/en/latest/miniconda.html]. Download and install the most recent release for your system. Version here doesn't matter as we'll be creating a custom Python 3.6 environment within Conda - yes, the flexibility is beautiful.

\subsubsection{Configure Python 3.6 Environment}
Upon successful installation of Miniconda, proceed with the following steps:

\begin{enumerate}
\item Start off by updating conda and then create a new environemnt with a specific version of Python - 3.6 in our case. Name your environment as you please and replace $your_env_name$ with your custom name. We used py36 to keep things simple.
\item Verify the installation of your environment:
\end{enumerate}

\subsection{MATLAB-Python Integration}
With the successful set up of MATLAB and Python, we need to ensure that the two languages can communicate with each other. Both languages provide tools to establish some form of communication between each other, and this section will guide you throught he step in ensuring that. It is advisable to perform all configuration via Terminal or command prompt as it minimizes any sources of errors in configuration.

\subsubsection{Configure Python}

\subsubsection{Configure MATLAB}

\subsection{Verify Bash Profile and Reboot}

\section{Study Configuration}

\end{document}